
\documentclass[11pt]{article}

% -------------------- Packages --------------------
\usepackage[margin=1in]{geometry}
\usepackage{amsmath, amssymb}
\usepackage{listings}
\usepackage{xcolor}
\usepackage{hyperref}
\usepackage{graphicx}
\usepackage{enumitem}
\usepackage{fancyhdr}
\usepackage[listings]{tcolorbox}
\usepackage{colortbl}
\usepackage{hyperref}   

% ---------- Modular row commands ----------
\newcommand{\memrow}[1]{\texttt{#1} \\ \hline}
\newcommand{\highlightmemrow}[1]{\cellcolor{gray!20}\texttt{#1} \\ \hline}

% ---------- Memory table environment ----------
\newenvironment{memorytable}[1][4cm]{
\begin{center}
\begin{tabular}{|p{#1}|}
\hline
}{
\end{tabular}
\end{center}
}

% -------------------- Header / Footer --------------------
\pagestyle{fancy}
\fancyhf{}
\lhead{ITI1121 -- Introduction To Computing}
\rhead{Lecture 8}
\cfoot{\thepage}

% -------------------- Code Styling --------------------
\lstset{
    basicstyle=\ttfamily\small,
    keywordstyle=\color{blue},
    commentstyle=\color{gray},
    stringstyle=\color{teal},
    numbers=left,
    numberstyle=\tiny,
    stepnumber=1,
    numbersep=8pt,
    frame=single,
    breaklines=true,
    tabsize=2
}

% -------------------- Custom Commands --------------------
\newcommand{\topic}[1]{\section*{#1}}
\newcommand{\subtopic}[1]{\subsection*{#1}}
\newcommand{\important}{\textbf{Important: }}
\newcommand{\defn}[1]{\textbf{Definition:} #1}



% -------------------- Document --------------------
\begin{document}
% -------------------- Title --------------------
\begin{center}
    {\Large \textbf{ITI1121 -- Introduction To Computing}}\\
    Lecture 8 -- Polymorphism with Abstract Classes (Comparable / Displayable)\\
    Date: February 5, 2026
\end{center}

\topic{Problem 1}
{Write a polymorphic method that sorts an array of objects}
\begin{lstlisting}[language=Java]
public static void sort2( int[] a ) {
    if ( a[ 0 ] > a[ 1 ] ) {
        int tmp = a[ 0 ];
        a[ 0 ] = a[ 1 ];
        a[ 1 ] = tmp;
    }
}
public static void sort2( Time[] a ) {
    if ( a[ 0 ].after( a[ 1 ] ) ) {
        Time tmp = a[ 0 ];
        a[ 0 ] = a[ 1 ];
        a[ 1 ] = tmp;
    }
}
public static void sort2( Shape[] a ) {
    if ( a[ 0 ].compareTo( a[ 1 ] ) > 0 ) {
        Shape tmp = a[ 0 ];
        a[ 0 ] = a[ 1 ];
        a[ 1 ] = tmp;
    }
}
\end{lstlisting}
\topic{Solution 1: Super-class Comparable}
\begin{lstlisting}[language=Java]
abstract class Comparable {
    public abstract int compareTo( Comparable other );
}
\end{lstlisting}
\begin{itemize}
    \item \important Element 1 is the keyword \texttt{abstract}. The implementation of \texttt{compareTo} depends on the particular type of object.
    \item In order to find the parameter type for \texttt{compareTo}, look at how \texttt{sort2} will call it.
\end{itemize}
\begin{lstlisting}[language=Java]
public class Array {
    public static void sort2( Comparable[] myArray ) {
        if ( myArray[ 0 ].compareTo( myArray[ 1 ] ) > 0 ) {
            Comparable tmp;
            tmp = myArray[ 0 ];
            myArray[ 0 ] = myArray[ 1 ];
            myArray[ 1 ] = tmp;
        }
    }
}
abstract class Comparable {
    public abstract int compareTo( Comparable other );
}
\end{lstlisting}
{Our first solution requires creating an abstract class called Comparable that
contains only one method, which is abstract, the method is called compareTo.
Any class who needs to use the (polymorphic) sort2 method must be a sub-class of
the class Comparable (and therefore must implement the method compareTo).
For instance, the classes Time and Shape would be modified so that both would
be subclasses of Comparable.
}

\subtopic{A first attempt: why it fails}
For the class \texttt{Time}, one might want to write:
\begin{lstlisting}[language=Java]
public class Time extends Comparable {
    ...
    public int compareTo( Comparable obj ) {
        int result;
        if ( timeInSeconds < obj.timeInSeconds ) {
            result = -1;
        } else if ( timeInSeconds == obj.timeInSeconds ) {
            result = 0;
        } else {
            result = 1;
        }
        return result;
    }
}
\end{lstlisting}
This produces a compilation error because \texttt{timeInSeconds} is not defined for the type \texttt{Comparable}.

\subtopic{Fix: downcasting inside compareTo}
To access \texttt{Time}-specific fields/methods, we cast the parameter back to \texttt{Time}:
\begin{lstlisting}[language=Java]
public class Time extends Comparable {
    ...
    public int compareTo( Comparable obj ) {
        Time other = (Time) obj;
        int result;
        if ( timeInSeconds < other.timeInSeconds ) {
            result = -1;
        } else if ( timeInSeconds == other.timeInSeconds ) {
            result = 0;
        } else {
            result = 1;
        }
        return result;
    }
}
\end{lstlisting}

Similarly, the class \texttt{Shape} can be changed so that it becomes a subclass of \texttt{Comparable} and implements \texttt{compareTo}:
\begin{lstlisting}[language=Java]
public class Shape extends Comparable {
    ...
    public int compareTo( Comparable o ) {
        Shape other = (Shape) o;
        int result;
        if ( area() < other.area() )
            result = -1;
        else if ( area() == other.area() )
            result = 0;
        else
            result = 1;
        return result;
    }
}
\end{lstlisting}

\subtopic{Polymorphic sorting method (2 elements)}
Here is a method that sorts any array of \emph{two} objects, as long as their class is a subclass of \texttt{Comparable}:
\begin{lstlisting}[language=Java]
public static void sort2( Comparable[] a ) {
    if ( a[ 0 ].compareTo( a[ 1 ] ) > 0 ) {
        Comparable tmp = a[ 0 ];
        a[ 0 ] = a[ 1 ];
        a[ 1 ] = tmp;
    }
}
\end{lstlisting}

\begin{itemize}
    \item \important This solution works, but it has a weakness: it relies on casting (e.g., \texttt{(Time) obj}).
    \item If you accidentally compare objects of different classes (e.g., a \texttt{Time} vs a \texttt{Shape}), the cast compiles but can fail at runtime with \texttt{ClassCastException}.
    \item (Later) Generics solve this cleanly by enforcing the correct type at compile time.
\end{itemize}

\subtopic{UML-style summary (conceptual)}
\begin{tcolorbox}[title=Class relationships, colback=white, colframe=black]
\begin{itemize}[leftmargin=*]
    \item \texttt{Comparable}: \texttt{+compareTo(o:Comparable): int}
    \item \texttt{Shape} extends \texttt{Comparable}
    \item \texttt{Circle} extends \texttt{Shape}
    \item \texttt{Rectangle} extends \texttt{Shape}
    \item \texttt{Time} extends \texttt{Comparable}
\end{itemize}
\end{tcolorbox}

\subtopic{Preview: the full fix (generics / Java library)}
\begin{itemize}
    \item \important In Java, there is already an interface \texttt{java.lang.Comparable<T>}.
    \item A type-safe version of this idea is:\\
    \texttt{public static <T extends Comparable<T>> void sort2(T[] a) \{ ... \}}
\end{itemize}

\topic{Problem 2}
Problem 2: write a (polymorphic) method that displays all the elements of an array
(arrays of shapes, buttons, balloons, etc.).

\begin{lstlisting}[language=Java]
class Graphics {
    public static void displayAll( Displayable[] myArray ) {
        for ( int i = 0; i < myArray.length; i++ ) {
            myArray[ i ].display();
        }
    }
}
\end{lstlisting}

\subtopic{Solution: inherit from an abstract class Displayable}
\begin{lstlisting}[language=Java]
public abstract class Displayable {
    public abstract void display();
}
\end{lstlisting}

\begin{itemize}
    \item Any class that should be displayable must extend \texttt{Displayable} and implement \texttt{display()}.
    \item \important This is the same polymorphism pattern as \texttt{Comparable}: the method body depends on the specific subclass.
\end{itemize}

\subtopic{Important facts / common pitfalls}
\begin{itemize}
    \item \important Upcasting is safe (subclass object in a superclass variable). Downcasting can fail at runtime.
    \item \important If a superclass reference points to a subclass object, method calls use the subclass override (dynamic dispatch).
    \item \important When an API must work for many unrelated classes, an \emph{interface} is often a better fit than forcing a shared superclass.
\end{itemize}

\end{document}

