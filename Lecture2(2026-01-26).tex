\documentclass[11pt]{article}

% -------------------- Packages --------------------
\usepackage[margin=1in]{geometry}
\usepackage{amsmath, amssymb}
\usepackage{listings}
\usepackage{xcolor}
\usepackage{hyperref}
\usepackage{graphicx}
\usepackage{enumitem}
\usepackage{fancyhdr}
\usepackage[listings]{tcolorbox}
\usepackage{colortbl}

% ---------- Modular row commands ----------
\newcommand{\memrow}[1]{\texttt{#1} \\ \hline}
\newcommand{\highlightmemrow}[1]{\cellcolor{gray!20}\texttt{#1} \\ \hline}

% ---------- Memory table environment ----------
\newenvironment{memorytable}[1][4cm]{
\begin{center}
\begin{tabular}{|p{#1}|}
\hline
}{
\end{tabular}
\end{center}
}

% -------------------- Header / Footer --------------------
\pagestyle{fancy}
\fancyhf{}
\lhead{ITI1121 -- Introduction To Computing}
\rhead{Lecture 4}
\cfoot{\thepage}

% -------------------- Code Styling --------------------
\lstset{
    basicstyle=\ttfamily\small,
    keywordstyle=\color{blue},
    commentstyle=\color{gray},
    stringstyle=\color{teal},
    numbers=left,
    numberstyle=\tiny,
    stepnumber=1,
    numbersep=8pt,
    frame=single,
    breaklines=true,
    tabsize=2
}

% -------------------- Custom Commands --------------------
\newcommand{\topic}[1]{\section*{#1}}
\newcommand{\subtopic}[1]{\subsection*{#1}}
\newcommand{\important}{\textbf{Important: }}
\newcommand{\defn}[1]{\textbf{Definition:} #1}



% -------------------- Document --------------------
\begin{document}

% -------------------- Title --------------------
\begin{center}
    {\Large \textbf{ITI1121 -- Introduction To Computing}}\\
    Lecture 4 -- OOP\\
    Date: January 26, 2025
\end{center}
\topic{Class}
What goes into a class:
\begin{itemize}
    \item  variables, methods, (nested) classes; (3)
    \item each belongs either to the class or the instance; (2)
    \item each can be public, package, (protected) or private; (4)
    \item each can be final or not. (2)
\end{itemize}
Declaring classes:
\begin{lstlisting}[language=Java]
public class PublicClass {
    public static int classVariable;
    ...
}
\end{lstlisting}
Declaring instance variables:
\begin{lstlisting}[language=Java]
public class PublicClass {
    public int instanceVariable;
    ...
}
\end{lstlisting}
% -------------------- Declaration Types --------------------
\topic{Declaration Types}
\begin{itemize}
    \item \textbf{public}: accessible from any other class. 
    \item \textbf{private}: accessible only within the declaring class. Use for encapsulation.
    \item \textbf{protected}: accessible within the same package and subclasses.
    \item \textbf{package-private (default)}: no modifier; accessible only within the same package.
    \item \textbf{static}: belongs to the class rather than an instance (class-level).
    \item \textbf{final}: value cannot change (for variables) or class/method cannot be extended/overridden.
    \item \textbf{abstract}: declares a method without implementation or a class that cannot be instantiated.
    \item \textbf{synchronized}: method or block that controls concurrent access in multi-threading.
    \item \textbf{transient}: field not serialized when using Java serialization.
    \item \textbf{volatile}: tells the JVM that a field may be modified by multiple threads; ensures visibility.
    \item \textbf{native}: indicates implementation is provided in platform-specific code (e.g., C/C++).
    \item \textbf{strictfp}: restricts floating-point calculations to ensure portability of floating-point results.
\end{itemize}
\topic {Ticket Example}
\begin{lstlisting}[language=Java]
public class Ticket {
    private staic int lastSerialNumber = 0; 
    private int serialNumber; 

    public Ticket() {
        this.serialNumber = lastSerialNumber; 
        lastSerialNumber++; //increments the last serial number
    }

    public int getSerialNumber() {
        return this.serialNumber;
    }

    public static int getLastSerialNumber() {
        return lastSerialNumber;
    }
}
\end{lstlisting}
\topic{Test file}
\begin{lstlisting}[language=Java]
public class Test {
    public static void main (String args[]) {

        System.out.println(Ticket.lastSerialNumber) //to call a static variable from a seperate class, class name is required.
        System.out.println(Ticket.getLastSerialNumber()); //getLastSerialNumber() is static therefore can be called without an initialized object .

        Ticket t1 = new Ticket(); //an instance is object created when its called with "new"
        Ticket t2 = new Ticket();
        Ticket t3 = new Ticket();

        System.out.println(t1.getSerialNumber);
        System.out.println(t2.getSerialNumber);
        System.out.println(t3.getSerialNumber);
    }
}
\end{lstlisting}
\topic{Instances, Static, etc}
\begin{itemize}
    \item In order to call a static variable, method, etc, from a seperate .java file, the class name is required.  
    \item Static (explained in declaration types) allows methods and variables to be called without initialization.
\end{itemize}

% -------------------- Private constructors --------------------
% -------------------- Constructors --------------------
\subtopic{Constructors}
Constructors run when an object is created with `new`. Rough JVM order when `new` is executed:
\begin{enumerate}
    \item Allocate memory on the heap for the new object (object header + fields).
    \item Set all instance fields to default values (0, false, null).
    \item Invoke the superclass constructor chain (an implicit call to `super()` is inserted if absent).
    \item Execute instance initializers and field initializers in textual order.
    \item Execute the body of the constructor (this is where you set field values, run validation, etc.).
    \item Return the reference to the newly constructed object (assigned to a local variable or field).
\end{enumerate}

Notes:
\begin{itemize}
    \item The reference variable (e.g. `Point p`) lives where it is declared (stack for local variables, heap if stored in another object), while the object data lives on the heap.
    \item The `this` reference inside the constructor points to the object being initialized.
    \item If no constructor is declared, the compiler provides a default no-arg constructor.
\end{itemize}

Example (default + parameterized):
\begin{lstlisting}[language=Java]
public class Point {
        int x, y;                    // fields are allocated and defaulted to 0
        public Point() { this(0,0); } // delegates to parameterized constructor
        public Point(int x, int y) {  // field initializers run before this body
                this.x = x;
                this.y = y;
        }
}
\end{lstlisting}

\subtopic{Private constructors}
Private constructors are used when you want to prevent callers from creating instances directly. Common cases:
\begin{itemize}
    \item \textbf{Utility / constant-holder classes:} prevent instantiation when all members are static.
\end{itemize}
\begin{lstlisting}[language=Java]
public final class Utils {
    private Utils() { throw new AssertionError("No instances."); }
    public static int add(int a, int b) { return a + b; }
}
\end{lstlisting}
\begin{itemize}
    \item \textbf{Singletons:} control instance creation (or use an enum singleton).
\end{itemize}
\begin{lstlisting}[language=Java]
public class Singleton {
    private static final Singleton INSTANCE = new Singleton();
    private Singleton() {}
    public static Singleton getInstance() { return INSTANCE; }
}
\end{lstlisting}
\begin{itemize}
    \item \textbf{Factory / builder enforcement:} force use of factory methods or builders for validation or different implementations.
\end{itemize}
\begin{lstlisting}[language=Java]
public class Person {
    private final String name;
    private Person(String name) { this.name = name; }
    public static Person of(String name) { return new Person(name); }
}
\end{lstlisting}

Notes: private constructors don't stop reflection by themselves; for singletons prefer enum-based singletons if serialization and reflection-safety matter.

\end{document}
