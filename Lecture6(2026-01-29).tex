
\documentclass[11pt]{article}

% -------------------- Packages --------------------
\usepackage[margin=1in]{geometry}
\usepackage{amsmath, amssymb}
\usepackage{listings}
\usepackage{xcolor}
\usepackage{hyperref}
\usepackage{graphicx}
\usepackage{enumitem}
\usepackage{fancyhdr}
\usepackage[listings]{tcolorbox}
\usepackage{colortbl}
\usepackage{hyperref}   

% ---------- Modular row commands ----------
\newcommand{\memrow}[1]{\texttt{#1} \\ \hline}
\newcommand{\highlightmemrow}[1]{\cellcolor{gray!20}\texttt{#1} \\ \hline}

% ---------- Memory table environment ----------
\newenvironment{memorytable}[1][4cm]{
\begin{center}
\begin{tabular}{|p{#1}|}
\hline
}{
\end{tabular}
\end{center}
}

% -------------------- Header / Footer --------------------
\pagestyle{fancy}
\fancyhf{}
\lhead{ITI1121 -- Introduction To Computing}
\rhead{Lecture 6}
\cfoot{\thepage}

% -------------------- Code Styling --------------------
\lstset{
    basicstyle=\ttfamily\small,
    keywordstyle=\color{blue},
    commentstyle=\color{gray},
    stringstyle=\color{teal},
    numbers=left,
    numberstyle=\tiny,
    stepnumber=1,
    numbersep=8pt,
    frame=single,
    breaklines=true,
    tabsize=2
}

% -------------------- Custom Commands --------------------
\newcommand{\topic}[1]{\section*{#1}}
\newcommand{\subtopic}[1]{\subsection*{#1}}
\newcommand{\important}{\textbf{Important: }}
\newcommand{\defn}[1]{\textbf{Definition:} #1}



% -------------------- Document --------------------
\begin{document}

% -------------------- Title --------------------
\begin{center}
    {\Large \textbf{ITI1121 -- Introduction To Computing}}\\
    Lecture 6 -- OOP and Inheritance\\
    Date: January 29, 2025
\end{center}
\topic{Local vs Instance}
\begin{lstlisting}[language=Java]
public class BankAccount {
    private double balance;

    //..

    public boolean transfer(BankAccount other, double amount) {

        if (this == other) {
            return false;
        }
        ...
    }
}
\end{lstlisting}
{Its important to optimize programs by using the right variable types. this, other and amount are all local variables. }
\begin{itemize}
    \item each object uses more memory than it needs to (takes local and instance variable memory)
    \item Gets into the way of the garbage collector.
\end{itemize}

\begin{lstlisting}[language=Java]
class A {

    int[] boxes;

    pubObjectIntoBoxes () {

        boxes = new int[1000];  
    }
}
\end{lstlisting}
\topic{Inheritance}
\begin{itemize}
    \item enables class heirarchy
    \item modular code and reusable code
\end{itemize}
{Subclass, child class, or derived class inherit traits from the parent/super class. \\eg Bird(Super) - Pidgeon(Sub)}
\begin{itemize}
    \item Java uses the "Is a" relationship expressed by "extends" in the class line.
\end{itemize}
\topic{What Gets Inherited}
{A class inherits all the characteristics (variables and methods) of its superclass(es)}
\begin{itemize}
    \item a subclass inherits all the methods and variables of its superclass(es);
    \item a subclass can introduce/add new methods and variables;
    \item a subclass can override the methods of its superclass(@Overide).
\end{itemize}
{Because of 2 and 3, the subclass is a specialization of the superclass, i.e. the
superclass is more general than its subclasses.}
\topic{Shape Class Example}
\begin{lstlisting}[language=Java]
public class ShapeTest {

    public static void main(String[] args) {
        Circle c = new Circle(5,5,10);
        Rectangle r = new Rectangle(10, 10, 10, 20);

        System.out.println(c.toString());
        System.out.println(r.toString());
    }
}
\end{lstlisting}
\begin{lstlisting}[language=Java]
public class Shape {

    protected double x; //visible to subclasses respective to subclasses in the same package.
    protected double y;

    public Shape() {
        x = 0;
        y = 0;
    }

    public Shape(double x, double y) {
        this.x = x;
        this.y = y;
    }

    public double getX() {
        return x;
    }

    public double getY() {
        return y;
    }

    public void moveTo(double x, double y) {
        this.x = x;
        this.y = y;
    }

    public String toString() {
        return "Located at (" + this.x + "," + this.y + ")";
    }
}   
\end{lstlisting}
\begin{lstlisting}[language=Java]
public class Circle extends Shape {

    private double radius;

    public Circle() {
        super(); //Calls the constructor of the immediate superclass
        radius = 0;
    }

    public Circle(double x, double y, double radius) {
        this.radius = radius;
    }

    public double getRadius() {
        return radius;
    }

    public String toString() {
        return "Circle " + super.toString() + " with radius " + radius;
    }

    
}
\end{lstlisting}
\begin{lstlisting}[language=Java]
public class Rectangle extends Shape {
    private double width;
    private double height;

    public Rectangle() {
        super();
        width = 0;
        height = 0;
    }

    public Rectangle(double x, double y, double width, double height) {
        super(x,y);
        this.width = width;
        this.height = height;

    }

    public double getWidth() {
        return width;
    }

    public double getHeight() {
        return height;

    }

    public String toString() {
        return "Rectangle " + super.toString() + "), width = " + width + " and height = " + height;
    }
}
\end{lstlisting}
\topic{Polymorphism}
{From the Greek words polus = many and morphˆe = forms, literally means has
many forms.}
\begin{itemize}
    \item polymorphism (overloading): a method name is associcated with
different blocs of code
    \item Inclusion (subtyping, data) polymorphism: an identifer (a reference variable) is associated with data of different types with the use of a subtype relation
\end{itemize}
\topic{Method Overloading}
{Method overloading means that two methods can have the same name but different signatures (the signature consists of the name and formal parameters of a method but not the return value).}
\end{document}