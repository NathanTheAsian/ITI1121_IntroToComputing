
\documentclass[11pt]{article}

% -------------------- Packages --------------------
\usepackage[margin=1in]{geometry}
\usepackage{amsmath, amssymb}
\usepackage{listings}
\usepackage{xcolor}
\usepackage{hyperref}
\usepackage{graphicx}
\usepackage{enumitem}
\usepackage{fancyhdr}
\usepackage[listings]{tcolorbox}
\usepackage{colortbl}
\usepackage{hyperref}   

% ---------- Modular row commands ----------
\newcommand{\memrow}[1]{\texttt{#1} \\ \hline}
\newcommand{\highlightmemrow}[1]{\cellcolor{gray!20}\texttt{#1} \\ \hline}

% ---------- Memory table environment ----------
\newenvironment{memorytable}[1][4cm]{
\begin{center}
\begin{tabular}{|p{#1}|}
\hline
}{
\end{tabular}
\end{center}
}

% -------------------- Header / Footer --------------------
\pagestyle{fancy}
\fancyhf{}
\lhead{ITI1121 -- Introduction To Computing}
\rhead{Lecture 7}
\cfoot{\thepage}

% -------------------- Code Styling --------------------
\lstset{
    basicstyle=\ttfamily\small,
    keywordstyle=\color{blue},
    commentstyle=\color{gray},
    stringstyle=\color{teal},
    numbers=left,
    numberstyle=\tiny,
    stepnumber=1,
    numbersep=8pt,
    frame=single,
    breaklines=true,
    tabsize=2
}

% -------------------- Custom Commands --------------------
\newcommand{\topic}[1]{\section*{#1}}
\newcommand{\subtopic}[1]{\subsection*{#1}}
\newcommand{\important}{\textbf{Important: }}
\newcommand{\defn}[1]{\textbf{Definition:} #1}



% -------------------- Document --------------------
\begin{document}

% -------------------- Title --------------------
\begin{center}
    {\Large \textbf{ITI1121 -- Introduction To Computing}}\\
    Lecture 7 -- OOP and Inheritance\\
    Date: Febuary 2, 2025
\end{center}
\topic{Method Overloading Cont..}
{Recall: Method overloading means that two methods can have the same name but
different signatures (the signature consists of the name and formal parameters of
a method but not the return value).} \\
\noindent{Constructors are often overloaded, this occurs for the class Shape:}
\begin{lstlisting}[language=Java]
Shape() {
    x = 0.0;
    y = 0.0;
}

Shape(int x, int y) { 
    this.x = x;
    this.y = y;
}
\end{lstlisting}
{Method overloading is sometimes referred as specific purpose polymorphism}
\topic{"True" Polymorphism: Motivation 1}
{Handling multiple datatype parameters for a specific method function.}
{example}
\begin{lstlisting}[language=Java]
public boolean isLeftOf( Circle c ) {
    return getX() < c.getX();
}
public boolean isLeftOf( Rectangle r ) {
    return getX() < r.getX();
}
\end{lstlisting}
{Clearly there is a problem with this, as every new shape requires a new method.}\\
{The solution for this is using the parent class as the parameter instead of the specific shape}
\begin{lstlisting}[language=Java]
public boolean isLeftOf( Shape a ){
    return getX() < s.getX();
}
\end{lstlisting}
\topic{Types}
{Assignment of a value of compile-time reference type S (source) to a variable of
compile-time reference type T (target) is checked as follows. \\ If S is a class type:}
\begin{itemize}
    \item If T is a class type, then S must either be the same class as T, or S must
be a subclass of T, or a compile-time error occurs
\end{itemize}
\begin{lstlisting}[language=Java]
Shape s;
Rectangle r;
r = new Rectangle(0,0,1,1);
s = r;
\end{lstlisting}
{This statement is valid, but "r = s" is not. Rectangle class cannot be coverted to Shape class. After "s == r" s turns to Rectangle class and therefore cannot be changed back into Shape class}
\topic{Abstraction}
\begin{itemize}
    \item Abstract classes cannot be initialized but can contain abstract methods AND regular methods
    \item Abstract methods can only be called in Abstract classes.
    \item A absttract class variable can stilled be initialized but not as an object
\end{itemize}
\begin{lstlisting}{language=Java}
public abstract class Shape {
    ...

    public abstract double area() { 
        ...
    }
}

Shape t; //compiles
Shape j = new Shape(); //will not compile as constructor is now a skeleton method.
\end{lstlisting}
{in the case:}
\begin{lstlisting}[language=Java]
Shape s;
s = new Rectangle();
s.area(); 
\end{lstlisting}
\begin{itemize}
    \item Java will called the overriden method from the Rectangle subclass in this case.
\end{itemize}
\topic{Interfaces}
{Interfaces are completely "abstract classes" in the sense that the methods are empty.}
\begin{itemize}
    \item Uses implements keyword.
    \item Are NOT used to create objects 
    \item Methods require @override in order for implementation
    \item No constructors
\end{itemize}
\end{document}

